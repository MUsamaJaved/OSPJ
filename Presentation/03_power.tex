\simplesection{Power-saving subsystem}
\lstset{
	language=C,
	basicstyle=\tiny,
	numbers=none,
	frame=none,
	tabsize=4,
	showtabs=false
}
	

\subsection{Power Management}
    
\frame{
    \frametitle{Power management approaches}
    \begin{itemize}
        \item Sleeping states (C-States)
        \item Core disabling.
        \item Dynamic frequency scaling (P-states)
    \end{itemize}
}

\frame{
    \frametitle{Speeping states \& C-States}
    \begin{itemize}
        \item What is a C-state?
        \item Requirements
        \item How to enable?  
        \item Fail and hunt for an approach
    \end{itemize}
}



\frame {
    \frametitle{Sleeping states. What is a C-state?}
    \begin{itemize}
        \item Definition
        \item The basic power states: C0, C1, C2, C3
        \item Architecture specific difference
    \end{itemize}
}

\frame {
    \frametitle{Sleeping states. Requirements}
    \begin{itemize}
        \item BIOS ACPI component
        \item ACPI module
        \item CPU idle driver
    \end{itemize}
}

\begin{frame}[fragile]
{ACPI module installation configuration}
\begin{lstlisting}[language=C,gobble=0]
CONFIG_PM=y
CONFIG_ACPI=y
CONFIG_ACPI_AC=y
CONFIG_ACPI_BATTERY=y
CONFIG_ACPI_BUTTON=y
CONFIG_ACPI_FAN=y
CONFIG_ACPI_PROCESSOR=y
CONFIG_ACPI_THERMAL=y
CONFIG_ACPI_BLACKLIRG_YEAR=0
CONFIG_ACPI_EC=y
CONFIG_ACPI_POWER=y
CONFIG_ACPI_SYSTEM=y
\end{lstlisting}
\end{frame}


\begin{frame}[fragile]
	{Sleeping states. How to enable C1?}
	\begin{itemize}
		\item	Meet the mentioned requirements
		\item	Read specific msr register - {\color{red}MSRC001\_0073}            
		\item[]	\begin{lstlisting}
/*
 * Reading the msr register.
 * @param msr_id - MSR Register to read
 * @return msr_value - The status of reading the register
 */
static long rdmsr_v(long msr_id){		
    long long msr_value;
    asm volatile ( "rdmsr" : "=A" (msr_value) : "c" (msr_id) );
    return msr_value;		
}
\end{lstlisting}
		\item	Or Execute {\color{red}HLT} instruction
	\end{itemize}
\end{frame}


\begin{frame}[fragile]{Fail and hunt for an approach}
	\begin{itemize}
		\item No ACPI configuration in BIOS even with the latest update
		\item Powertop didn't detect available C-states
		\item Specific CPU idle driver absent for our CPU
		\item Approach fail, and hunting green piece energy mode
	\end{itemize}
\end{frame}

\begin{frame}[fragile]{Core disabling}
	\begin{enumerate}
		\item Advantages of the approach. 
		\item[] \lstinline$sudo sh -c 'echo 0 > /sys/devices/system/cpu/cpu1/online'$
		\item Benchmark
	\end{enumerate}
\end{frame}

\frame{
    \frametitle{Benchmark Core disabling vs DFS}
    \begin{table}[h]
        \begin{tabular}{ | l | c | c | c | c|}
    \hline
        Experiments & Cores & Freq      & Energy    & Response          \\ 
        \hline
        1 Baseline  & 4     & 3 GHz     & 77194 Ws  & 0                 \\ 
        \hline
        2           & 3     & 3 GHz     & 71693 Ws  & 15-45ms           \\
        \hline
        3           & 4     & 3 cores on& 73150 Ws  & 5μs               \\
        &       & 3 GHz and &           &                   \\
        &       & 1 core on &           &                   \\
        &       & 800 MHz   &           &                   \\
        \hline
        4           & 4     & 800 MHz   & 37966 Ws  & 5μs               \\
        \hline
        \end{tabular}
    \caption*{All experiments were performed with a cpu-load of 100\%}
    \end{table}
}


\begin{frame}{P-states}
    \begin{enumerate}
        \item With P-states, frequencies can be changed
        \item ACPI-cpufreq allows P-States of AMD Phenom(tm) II X4 940 Processor:
    \end{enumerate}
        \begin{table}[h]
        \begin{tabular}{ | l | c | c |}
    \hline
    P-state & Frequency \\
    \hline
    P0 & 3.0Ghz \\
    P1 & 2.3Ghz \\
    P2 & 1.8Ghz \\
    P3 & 800Mhz \\
    \hline
     \end{tabular}
    \end{table}
\end{frame}

\begin{frame}[fragile]{Ways to switch P-states}
	\begin{itemize}
		\item	Via shell:\\
			\lstinline$cpupower -c 0 frequency-set -f 800000$
		\item	Via shell with sysfs:
			\begin{lstlisting}
sudo sh -c 'echo userspace >
    /sys/devices/system/cpu/cpu0/cpufreq/scaling_governor'
sudo sh -c 'echo 800000 >
    /sys/devices/system/cpu/cpu0/cpufreq/scaling_setspeed'
\end{lstlisting}
		\item	Via c-function:
			\begin{enumerate} 
				\item from \textbf{user}-space with sysfs:\\
					\lstinline$sysfs_set_frequency(unsigned int cpu, unsigned long target_frequency);$
				\item from \textbf{kernel}-space
			\end{enumerate}
	\end{itemize}
\end{frame}

\begin{frame}{P-state-switching from kernel-space}
	\begin{itemize}
		\item interface not clearly defined (yet?) %Comment: under development, mention big load of messages in linux-pm developer-mailing-list
		\item cpufreq\_policy (struct defined in include/linux/cpufreq.h)
		\item Tried to get current policy, manipulate and set it again\\
			$\rightarrow$ But failed
	\end{itemize}
\end{frame}

\begin{frame}[fragile]{Code example P-states}
	\begin{lstlisting}
struct cpufreq_policy *policy, *policy_max, *policy_min;
struct cpufreq_policy *cpu_policy = NULL;
int tmpGetPolicyResult;
int error;
policy = (struct cpufreq_policy*) kmalloc(sizeof (struct cpufreq_policy), GFP_ATOMIC);
policy_max = (struct cpufreq_policy*) kmalloc(sizeof (struct cpufreq_policy), GFP_ATOMIC);
policy_min = (struct cpufreq_policy*) kmalloc(sizeof (struct cpufreq_policy), GFP_ATOMIC);
if (!(policy && policy_max && policy_min))
    /* error handling */
error = cpufreq_get_policy(policy, cpuNo); 
if (!error && policy)
    /* error handling */
memcpy((void*)policy_min, (void*)policy, sizeof(struct cpufreq_policy));
memcpy((void*)policy_max, (void*)policy, sizeof(struct cpufreq_policy));
policy_max->min = 3000000;
policy_max->max = 3000000;
policy_min->min =  800000;
policy_min->max =  800000;
cpufreq_cpu_put(policy_max);
kfree(policy);
policy = NULL;
kfree(policy_min);
policy_min = NULL;
kfree(policy_max);
policy_max = NULL;
\end{lstlisting}
\end{frame}

\begin{frame}{Plan for P-state-switching from kernel-space}
    \begin{itemize}
		\item a governor has a function-pointer store\_setspeed:
		\item \lstinline$int (*store_setspeed) (struct cpufreq policy *policy, unsigned int freq);$
		\item hand over our policy and frequency
		\item fields for min-, max- and cur frequency
    \end{itemize}
\end{frame}
